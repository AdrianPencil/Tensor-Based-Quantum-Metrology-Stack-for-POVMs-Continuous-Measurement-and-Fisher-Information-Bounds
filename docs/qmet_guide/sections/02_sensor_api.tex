\section{Sensor API}

Sensor models produce quantum states or expectations as functions of time and parameters.

\subsection{\code{sensor/sensor\_model.py}}

\begin{itemize}
  \item \code{QubitSensorModel}: closed-form qubit evolution under a \(Z\) Hamiltonian with optional pure dephasing.
  \item Pauli matrices \(\sigma_x,\sigma_y,\sigma_z\) are provided for convenience.
\end{itemize}

The core method is:
\[
\rho(t) = U(t)\rho(0)U(t)^\dagger,\qquad
U(t)=\exp\!\left(-\ii \frac{\omega t}{2}\sigma_z\right),
\]
with an optional phenomenological dephasing that damps off-diagonal terms in the \(\sigma_z\) basis.

\subsection{\code{sensor/device\_response.py}}

A device response maps a sensor-level scalar \(x\) to a measured unit \(y\).
The initial model is affine:
\[
y = g x + b,
\]
with optional clipping for saturation.

\subsection{\code{sensor/calibration.py}}

Calibration fits \(g,b\) from reference pairs \((x_{\text{true}}, y_{\text{meas}})\) using least squares.
It also provides an inverse map to convert measurements back to sensor units.
