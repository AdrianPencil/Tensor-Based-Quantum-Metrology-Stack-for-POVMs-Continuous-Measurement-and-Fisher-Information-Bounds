\section{Measurement models}

Measurement models connect quantum states to observed data.

\subsection{Discrete POVMs}

A POVM is a set \(\{E_k\}\) such that:
\[
E_k \succeq 0,\qquad \sum_k E_k = \I.
\]
Given \(\rho\), the outcome probabilities are:
\[
p_k = \Tr(E_k \rho).
\]

The code provides:
\begin{itemize}
  \item validation of completeness
  \item probability evaluation via vectorized traces
  \item sampling outcomes given an RNG
\end{itemize}

\subsection{Continuous diffusive measurement record}

For a Hermitian observable \(O\), a standard diffusive record model is:
\[
\dd Y = m(t)\dd t + \dd W,
\qquad
m(t)=2\sqrt{\eta\kappa}\,\langle O\rangle_{\rho(t)},
\]
where \(\kappa>0\) is measurement strength and \(\eta\in[0,1]\) is efficiency.

The module \code{measurement/continuous.py} provides the record model, while
\code{measurement/backaction.py} updates the state.

\subsection{Backaction update}

The implemented Ito step is:
\[
\dd\rho = \kappa \mathcal{D}[O](\rho)\dd t + \sqrt{\eta\kappa}\,\mathcal{H}[O](\rho)\dd W,
\]
with
\[
\mathcal{H}[O](\rho)= O\rho + \rho O - 2\langle O\rangle_\rho \rho.
\]

After each step the code projects back to a clean density matrix representation:
Hermitian, trace one, and numerically PSD.

\subsection{Readout noise}

A basic readout layer is additive Gaussian noise:
\[
y = \mu + \epsilon,\qquad \epsilon\sim \mathcal{N}(0,\sigma^2),
\]
with vectorized log-likelihood used throughout estimation and tests.
