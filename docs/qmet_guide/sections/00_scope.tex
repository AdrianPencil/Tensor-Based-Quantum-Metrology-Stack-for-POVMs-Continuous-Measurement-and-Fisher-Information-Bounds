\section{Scope}

\qmet{} is a small, test-backed library for composing quantum metrology problems:
\[
\text{sensor model} \rightarrow \text{measurement model} \rightarrow \text{estimator} \rightarrow \text{metrics}.
\]

\subsection{Design constraints}

\begin{itemize}
  \item Minimal and explicit math interfaces.
  \item Small number of primitives per file.
  \item Vectorized NumPy kernels where practical.
  \item Deterministic results under a seed.
\end{itemize}

\subsection{What is in scope now}

\begin{itemize}
  \item Qubit-level protocol models (Ramsey, echo, interferometric fringes).
  \item Discrete POVM probabilities and sampling.
  \item Diffusive continuous measurement record model and SME backaction step.
  \item Gaussian readout likelihood, Fisher information, CRB.
  \item Noise characterization tools (Welch PSD, overlapping Allan deviation).
\end{itemize}

\subsection{What is intentionally not in scope now}

\begin{itemize}
  \item Full device Hamiltonians or ab initio material models.
  \item Large-scale many-body simulation or tensor network backends.
  \item Automatic differentiation pipelines.
\end{itemize}
